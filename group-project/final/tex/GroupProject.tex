\documentclass[12pt]{article}

% file admin
\newcommand{\bibfile}{GroupProject}

\newcommand{\theauthor}{Shreya Ghosh, Paul Landes, Nianzu Ma, Pavan Reddy}
\newcommand{\doctitle}{Machine Learning Final Report Spring 2014}
\newcommand{\doctype}{Capital BikeShare Data Set Analysis and Prediction}
\newcommand{\theuniversity}{University of Illinois at Chicago}

\newcommand{\ci}[1]{\cite{#1}}
\newcommand{\rtt}{${\mathit R}^2$}
\newcommand{\figwidth}{5.5in}

\usepackage{zenacademic}
\date{}

\author{\theauthor}

\begin{document}

\maketitle
\thispagestyle{empty}
\newpage
%\vspace{-1cm}
\tableofcontents
\newpage

\ddsec{intro}{Introduction}

Bike sharing systems have made renting bikes efficient and quick with
memberships, multiple bike locations and easy rental and return
process. Through these systems, a user is able to easily rent a bike from a
particular station and return back at another station. The biggest issue bike
sharing systems face is due to adverse effects of harsh weather conditions on
biking. This paper shows how different weather conditions would affect bike
usage. The data used here has log of bikes rented in different seasons, under
different weather conditions and on weekdays or holidays. We use different
regression and classification algorithms to make relevant predictions for bike
sharing systems, which would help them deal with extreme conditions like very
few bikes were rented or all of the bikes were rented.


\ddsubsec{book}{Instructional Resources}

The class machine learning assigned book\ci{murphy2012machine} was used to
formulate the theory behind the analysis and machine learning tasks.


\ddsubsec{data}{Data}

In recent years bikes have turned out to be an efficient alternate mode of
transportation in some cities, which lead to growth of bike-sharing
systems. These systems are quite convenient for users, since the user do not
have to worry about parking, they have options where you can become a member
and often these systems let first 30-60 minutes of ride free. Currently there
are over 500 bike-sharing programs around the world which is composed of over
500 thousands bicycles\ci{ucibikeshare}.

The data set being used in this paper was created from trip logs kept by
Capital Bikeshare (CaBi) in Washington, DC, where Capital Bikeshare is
currently the largest in the nation with over 1,200 bicycles at 140
stations. The data set put together by The University of California at Irvine
has features like date, weather, weekdays, holidays and count of bikes
rented. These features of the data make it apt for research such as finding
patterns between different features and number of bikes rented.  Feature
selection was in part chosen based on data manipulation\ci{gebhart2013impact}.

The data set was created from three different sources. Bike trip logs were
collected from Capital Bikeshare, merged with weather data from Freemeteo and a
holiday schedule provided by The District of Columbia. The data set being used
in this paper has hourly as well as daily usage data over the course of two
years (2011 - 2012).
% The data was collected from Washington, DC and since
%Washington observes average windspeed throughout the year, we choose to ignore
%windspeed as a feature in few analysis models.


\ddsubsec{data-def}{Data Definition} 

The compiled set of parameters features used follow in \ddtabref{params}.

\begin{ddbasictable}{params}{Data Features}{|l|l|}
\hline
{\bf Variable} & {\bf Description} \\ \hline
instant & record index \\ \hline
dteday  & date \\ \hline
season  & season (1:springer, 2:summer, 3:fall, 4:winter) \\ \hline
yr  & year (0: 2011, 1:2012) \\ \hline
mnth  & month ( 1 to 12) \\ \hline
hr  & hour (0 to 23) \\ \hline
holiday  & weather day is holiday or not\footnote{extracted from http://dchr.dc.gov/page/holiday-schedule} \\ \hline
weekday  & day of the week \\ \hline
workingday  & if day is neither weekend nor holiday is 1, otherwise 0. \\ \hline
weathersit  & intensity of weather (1 - 4) \\ \hline
temp  & Normalized temperature in Celsius. The values are divided to 41 (max). \\ \hline
atemp & Normalized feeling temperature in Celsius. The values are divided to
(max) \\ \hline
hum & Normalized humidity. The values are divided to 100 (max) \\ \hline
windspeed & Normalized wind speed. The values are divided to 67 (max) \\ \hline
casual & count of casual users \\ \hline
registered & count of registered users \\ \hline
cnt & count of total rental bikes including both casual and registered \\ \hline
\end{ddbasictable}


\ddsec{prob}{Problem Definition}

The data has hourly and daily usage. In this paper we develop an understanding
about how humidity, temperature, rainfall, snow and other weather conditions
affect the number of bikes being rented. We also determine bike rental patterns
of different kind of users\footnote{Registered and non-registered users}. This
analysis would help bike sharing companies to predict what number to bikes they
need to stock on any given day.

\ddsec{feat-sel}{Feature Selection}

We used various methods for data processing and choosing features to build the
classification and regression models.

The majority of the data instances of the hourly bike rental count are in the range of 0 - 400
as shown in \ddfigref{freq-histogram}.

\ddfigure{\figwidth}{freq-histogram}{Bike Rental Count Frequency}


\ddsubsec{feat-scaling}{Feature Scaling}

Feature Scaling is a method used for data processing to standardize the range
of the feature variables. The range of the features are normalized to a value
having unit variance. The values can standardized to either ranging from (0 to
1), this can be obtained by normalizing by the maximum value or ranging from
(-1 to 1) which can be obtained by subtracting the value by the mean and then
normalizing by the maximum value.

For example, the temperature ranges from 0 to 40 deg C but it is standardized
to a variable having unit variance ranging from (0 to 1). We used standardized
features for variables temperature, feels like temperature, humidity and wind
speeds. The consequence of using feature scaling is to make sure all the
features have similar variances and ranges which inturn helps in faster
convergence of gradient descent of the cost function for the regression models.

\ddsubsec{dep-cor-mat}{Dependent Variable Correlation matrix}

\begin{ddbasictable}{feat-cor-mat}{Features Correlation Matrix}{|l|l|l|l|l|l|l|l|l|}
\hline
& \bf{workingday} & \bf{temp} & \bf{atemp} & \bf{hum} & \bf{windspeed} & \bf{cnt} \\ \hline
\bf{workingday} & 1.00 & 0.06 & 0.05 & 0.02 & -0.01 & 0.03 \\ \hline
\bf{temp} & 0.06 & 1.00 & 0.99 & -0.07 & -0.02 & 0.40 \\ \hline
\bf{atemp} & 0.05 & 0.99 & 1.00 & -0.05 & -0.06 & 0.40 \\ \hline
\bf{hum} & 0.02 & -0.07 & -0.05 & 1.00 & -0.29 & -0.32 \\ \hline
\bf{windspeed} & -0.01 & -0.02 & -0.06 & -0.29 & 1.00 & 0.09 \\ \hline
\bf{casual} & -0.30 & 0.46 & 0.45 & -0.35 & 0.09 & 0.69 \\ \hline
\bf{registered} & 0.13 & 0.34 & 0.33 & -0.27 & 0.08 & 0.97 \\ \hline
\bf{cnt} & 0.03 & 0.40 & 0.40 & -0.32 & 0.09 & 1.00 \\ \hline
\end{ddbasictable}

Interpretation of \ddtabref{feat-cor-mat} are given below:
\begin{itemize}
\item 0 indicates no linear relationship.
\item +1 indicates a perfect positive linear relationship: as one variable increases in its values, the other variable also increases in its values via an exact linear rule.
\item -1 indicates a perfect negative linear relationship: as one variable increases in its values, the other variable decreases in its values via an exact linear rule.
\item Values between 0 and 0.3 (0 and -0.3) indicate a weak positive (negative) linear relationship via a shaky linear rule.
\item Values between 0.3 and 0.7 (0.3 and -0.7) indicate a moderate positive (negative) linear relationship via a fuzzy-firm linear rule.
\item Values between 0.7 and 1.0 (-0.7 and -1.0) indicate a strong positive (negative) linear relationship via a firm linear rule.
\end{itemize}

According to the values in \ddtabref{feat-cor-mat} only temp (normalized
temperature, value = 0.4), atemp (normalized feeling temperature, value = 0.4),
and humidity (value = -0.32) has moderate positive-negative linear relationship
with count (the total bike count of each hour). This indicates that these
features implies an important role in regression and classification algorithms.
This information helps us to do feature selection when building models for
prediction.  \ddFigref{temp-count-cor} shows the relationsip between temperature and bike rental
count.

\ddfigure{\figwidth}{temp-count-cor}{Temperature Bike Counts Correlation}

Note that, the correlation coefficient of windspeed and count is only 0.09.
This number indicates a very weak linear relationship correlation.  This
challenges the common sense of those living in colder climates as windspeed,
and therefore comfort of the ride, would effect the willingness commute by
bike.  However, the average in Washington D.C. the value is 8.2 MPH, this can
be described as a ``gentle breeze'' defined by the
Beaufort\footnote{http://en.wikipedia.org/wiki/Beaufort\_scale} wind force
scale. Therefore, both the correlation coefficient value and real windspeed
data show that windspeed will not affect the rental numbers of the D.C. area.

\ddFigref{wind-count-cor} shows the correlation between wind speeds and
rentals, which resulted in $0.09$.

\ddfigure{\figwidth}{wind-count-cor}{Correlation of Wind Speed and Rentals}


\ddsubsec{stepwise}{Stepwise Regression Feature Selection}

%% no edits from paul after this 4/27/2014

Stepwise regression is a method used to automate the process of choosing the
feature variables used in the regression models. This method allows selecting a
subset of the predictor variables from a larger set of predictor variables by
performing a stepwise regression. The stepwise regression can be performed by
forward selection, backward elimination or both. Forward selection is a method
where the algorithm starts with no predictor variables and adds variables
iteratively, choosing variable based on which improves the model the
most. Backward elimination is where the algorithm starts with a model that has
all the predictor variables, and then deletes variables from the model that
improves the model the most after its deletion, iteratively. We used both
forward selection and backward elimination stepwise regression.

Using stepwise regression, we were able to ascertain that features such a
hour23 (23rd hour of the day), season\_winter (winter season) are eliminated
from the model. The predictor variables such as hour23 and season\_winter have
very low correlation coefficients, -0.1171 and 0.02942 respectively.


\ddsec{approach}{Approach}

Our approach involved linear regression, logistical regression, na{\"i}ve Bayesian
classification, SVM, and decision tree.  Both linear regression and SVM was used
for prediction while logistical regression, na{\"i}ve Bayesian classification,
SVM and decision tree was used for classification.


\ddsubsec{reg}{Regression}

\ddsubsubsec{lin-reg}{Linear Regression}

We built multiple linear regression models for both daily and hourly data. We
built multiple regression models based on different feature selections.  Based
on the data procession, predictor variable analysis, feature selection methods,
we built multiple linear regression models, which takes the general form:
\begin{equation}
y_i = \beta_0 + \beta_1x_{i1} + \cdots + \beta_px_{ip} + \varepsilon_i = {\mathbf
x}_i^{T}\beta + \varepsilon_i, \; \; \; i = 1, \ldots, n
\end{equation}

where $p$ is the number of variables. Examples of values of x are weather
(i.e. rain), temperature, number of rentals, etc. The $\beta$ values are the
learned parameters.


\ddsubsubsec{non-lin-reg}{Non-linear Regression}

Based on the analysis of the predictor variables, we build non linear
regression models using nonlinear combination of the model parameters. We
transformed predictor variables such as the temperature, feels like temperature
to their respective square values. Predictor variables such as wind speed which
have lower impact on the model, based on the correlation analysis of predictor
variables, were transformed to square roots of the respective values.  Using
these non linear combination of the predictor variables, we built non linear
regression models for the hourly dataset.


\ddsubsubsec{svm}{Support Vector Machines}

Support Vector Machines (SVM) is a supervised machine learning method that can
be used to build regression models.  We used SVM to build regression models for
the hourly dataset. We tuned the SVM models with a multitude of parameters such
as kernel functions, degree of the polynomial, constant for regularization term
which intuitively is a parameter that helps mitigate the effects of
overfitting, $\gamma$ the parameter that defines how far a training example's
influence reaches etc. For kernel functions we experimented with SVM models
that used different kernel functions such as linear, polynomial, radial and
sigmoid functions.


\ddsubsec{reg-class}{Classification}

For bike sharing companies, the cost of arranging bikes to different station to
meet customers’ needs is a big part of the total operation cost. The prediction
given by regression model could provide good reference for such
companies. However, there are some extreme data instances that cannot be solved
by regression model prediction. These case will make regression approach less
reliable and will mislead companies decisions for arranging bikes for
stations. If in some cases, the number of bike needed is extreme high, but
regression model cannot give such prediction, then it would be a huge loss of
income of the company and also harm the customers benefit.  A reliable
classification approach is need to detect such extreme cases and let companies
to prepare for such cases.

We will discretized bike count as ``low'' and ``high'' to show the extreme
case. But for different companies, the ability of the amount of bikes to
prepare for stations at each hour is quite different. It means the definition
of ``low'' and ``high'' of each bike sharing company is different. We will set
the boundary of ``high'' as 600, 700, 800 respectively and set the
boundary of ``low'' as 100 to analyze the performance of our classification
model.

We used J48 decision tree, naive bayes, SVM and logistic regression models to
classify hourly bike sharing data. All these models were used with combination
of multiple features trying to analyze how these features affected prediction
outcome. Using the correlation matrix we decided which combination of features
to be used in classification models.


\ddsubsec{eval}{Evaluation}

The results of the experiments gave a wide variety of success.

\ddsubsubsec{train-test}{Training and Testing}

We split the entire dataset into a training and test set in 70 : 30 percent
ratio. The models were trained on the training set and we used these model to
predict the total number of bike rentals in the testing set.

\ddsubsubsec{kfold}{K Fold Cross Validation}

We are using 10 fold cross validation to test accuracy of our models. It
divides the data in 10 sets and treats 9 sets for training, 1 set for
testing. This process is repeated 10 times and the output represents average of
all the runs.

\ddsubsubsec{reg}{Regression}

The regression model and predictions are evaluated using Coefficient of
Determination or \rtt\ and Root Mean Squared Error (RMSE).  \rtt\ is the
measure of how well a model has fit the data. \rtt\ can be computed by
\begin{equation}
\begin{gathered}
R^2 \equiv 1 - \frac{SS_{res}}{SS_{tot}}\\
SS_{tot} = \sum_i(y_i - \overline{y})^2
\end{gathered}
\end{equation}
where $\overline{y}$ is the mean value; that is the total sum of squares and
$SS_{res}$ is the sum of squared residuals.

Root Mean Squared Error (RMSE) is a measure of the square root of the mean of
the squared errors or squared deviations of the prediction values from the
actual values. This value is an indicator to compare different models. A lower
RMSE value indicates a better model.  RMSE can also be used as a measure of how
well a model fits the data and if the model is overfit or underfit. This is
done by computing the RMSE of both the training set and the testing set. A
model that is fit well will have comparable RMSE values. But, if the training
RMSE is significantly lower than the RMSE of testing set then it indicates a
model that is overfit, in which case we can use methods like regularization or
increase and decrease the values of regularization parameters accordingly.


\ddsubsubsec{eval-class}{Classification}
            
Evaluation for J48 and logistic regression models is done based on accuracy,
precision, recall and f-measure. These measures can be computed from truth
table. Accuracy shows how close the model was in predicting right outcomes,
precision is the fraction of data marked as relevant and relevant data by test
$\frac{TP}{TP + FP}$, recall is the fraction of data marked as relevant and all
data supposed as relevant $\frac{TP}{TP + FN}$ and f-measure is another measure
to compute a model’s accuracy using precision and recall.



\ddsec{results}{Experimental Results}

\ddsubsec{reg-res}{Regression}

\ddsubsubsec{lin-reg-res}{Linear and Non-Linear Regression}

The R-squared training data value obtained from the linear regression runs was
0.2522 and the coefficients detailed in table \ddtabref{lin-reg-res}.  The RMSE
of the predicted values is 207.856.  These analysis comes from the hourly data.

The processed data mentioned in \ddsecref{feat-sel} yielded an R-squared
training data value of 0.646.  The RMSE of the predicted values is 163.3292.

This test was repeated by square rooting humidity and windspeed, squaring atemp
and yielded an R-squared value of 0.6442.  These variables were chosen from the
data analysis performed on the correlation matrix.  The RMSE of the predicted
values is 163.8344.


\begin{ddbasictable}{lin-reg-res}{Linear Regression Coefficients by Hour}{|c|l|l|}
\hline
{\bf Parameter} & {\bf Description} & {\bf Coefficient} \\ \hline
$\beta_0$ & Intercept & 123.7933 \\ \hline
$\beta_1$ & season & 7.6740 \\ \hline
$\beta_2$ & mnth & -2.3836 \\ \hline
$\beta_3$ & holiday & -23.3311 \\ \hline
$\beta_4$ & weekday & 1.3178 \\ \hline
$\beta_5$ & workingday & -0.5981 \\ \hline
$\beta_6$ & weathersit & 1.6828 \\ \hline
$\beta_7$ & atemp & 357.3604 \\ \hline
$\beta_8$ & hum & -225.3344 \\ \hline
$\beta_9$ & windspeed & 36.6928 \\ \hline
\end{ddbasictable}


\ddsubsubsec{svm-res}{SVM Regression}

The stepwise regression given in \ddsecref{stepwise} did not provide good
results.  Instead the following parameters were used for the SVM model.

\begin{packedlist}
\item discretized seasons
\item discretized months
\item discretized hours
\item discretized weekdays
\item workingday
\item discretized weather (clear, cloudy, perception)
\item atemp
\item humidity
\item windspeed
\end{packedlist}

However, it did perform similar to linear regression as described in
\ddsecref{lin-reg-res} with a RMSE value of 163.1421.

The epsilon value used for all SVM regression runs was 0.1.  The results for
the cross fold validation for SVM are given in \ddtabref{svm-results}.

\begin{ddbasictable}{svm-results}{SVM Results}{|l|l|l|l|l|l|}
\hline
{\bf Cost} & {\bf Gamma} & {\bf Kernel} & {\bf Degree} & {\bf Vector Count} & {\bf RMSE} \\ \hline
1000 & 0.0001 & radial & N/A & 8724 & 139.6982 \\ \hline
10 & 0.01 & sigmoid & N/A & 12963 & 1427.655 \\ \hline
10 & 0.01 & polynomial & 2 & 8807 & 133.0355 \\ \hline
10 & 0.01 & polynomial & 3 & 8317 & 128.6375 \\ \hline
10 & 0.01 & polynomial & 4 & 8345 & 129.9472 \\ \hline
\end{ddbasictable}

The scatter plot of the SVM prediction with the bike rental counts in
\ddfigref{actual-vs-predict-svm} graphically shows the error well.

\ddfigure{\figwidth}{actual-vs-predict-svm}{Actual Rentals vs. Predicated Values}


\ddsubsec{class-res}{Classification}

\ddsubsubsec{des-tree-res}{J48 Decision Trees}

\begin{ddbasictable}{dec-tree-res}{Decision Tree Results}{|l|l|l|l|l|l|l|}
\hline
& {\bf TP Rate} & {\bf FP Rate} & {\bf Precision} & {\bf Recall} & {\bf F-Measure} & {\bf Class} \\ \hline
& 0.999 & 0.079 & 0.997 & 0.999 & 0.998 & low \\ \hline
& 0.921 & 0.001 & 0.976 & 0.921 & 0.948 & high \\ \hline
Weighted Avg. & 0.996 & 0.075  & 0.996 & 0.996 & 0.996 &\\ \hline
\end{ddbasictable}

\begin{ddbasictable}{dec-tree-res}{Decision Tree Confusion Matrix}{|l|l|l|}
\hline
{\bf Low} & {\bf High} & \\ \hline
2064 & 2 & Low \\ \hline
7 & 82 & High \\ \hline
\end{ddbasictable}

The accuracy of the decision tree classification was 99.536\% 


\ddsubsubsec{log-reg-res}{Logistic Regression}

\begin{ddbasictable}{log-reg-res}{Logistic Regression Results}{|l|l|l|l|l|l|l|}
\hline
& {\bf TP Rate} & {\bf FP Rate} & {\bf Precision} & {\bf Recall} & {\bf F-Measure} & {\bf Class} \\ \hline
& 0.998 & 0.047 & 0.998 & 0.998 & 0.998 & low \\ \hline
& 0.953 & 0.002 & 0.956 & 0.953 & 0.955 & high \\ \hline
Weighted Avg. & 0.996 & 0.045 & 0.996 & 0.996 & 0.996 & \\ \hline

\end{ddbasictable}

\begin{ddbasictable}{dec-tree-res}{Logistic Regression Confusion Matrix}{|l|l|l|}
\hline
{\bf Low} & {\bf High} & \\ \hline
6996 & 14 & Low \\ \hline
15  & 307 & High \\ \hline
\end{ddbasictable}

The accuracy of logistic regression classification was 99.60\%.


\ddsubsubsec{svm-class-res}{SVM (SMO)}

\begin{ddbasictable}{svm-class-res}{SVM Classification By Hour}{|l|l|l|l|l|l|l|}
\hline
& {\bf TP Rate} & {\bf FP Rate} & {\bf Precision} & {\bf Recall} & {\bf F-Measure} & {\bf Class} \\ \hline
0.999 & 0.069 & 0.997 & 0.999 & 0.998 & 0.61 & low \\ \hline
0.931 & 0.001 & 0.979 & 0.931 & 0.955 & 0.906 & high \\ \hline
Weighted Avg. & 0.996 & 0.066 & 0.996 & 0.996 & 0.996 & 0.623 \\ \hline
\end{ddbasictable}


\begin{ddbasictable}{svm-conf-res}{SVM Classification Confusion Matrix}{|l|l|l|}
\hline
{\bf Low} & {\bf High} & \\ \hline
2339 & 2 & Low \\ \hline
7 & 95 & High \\ \hline
\end{ddbasictable}

The accuracy of SVM classification was 99.6316\%.


\ddsubsubsec{bn-res}{Na{\"i}ve Bayes}

\begin{ddbasictable}{svm-class-res}{Na{\"i}ve Bayes Classification By Hour}{|l|l|l|l|l|l|l|}
\hline
& {\bf TP Rate} & {\bf FP Rate} & {\bf Precision} & {\bf Recall} & {\bf F-Measure} & {\bf Class} \\ \hline
0.998 & 0.118 & 0.995 & 0.998 & 0.997 & 0.878 & low \\ \hline
0.882 & 0.002 & 0.957 & 0.882 & 0.918 & 0.958 & high \\ \hline
Weighted Avg. & 0.993 & 0.113 & 0.993 & 0.993 & 0.993 & 0.881 \\ \hline
\end{ddbasictable}

\begin{ddbasictable}{nb-res}{Na{\"i}ve Bayes Classification By Hour}{|l|l|l|l}
\hline
{\bf Low} & {\bf High} & \\ \hline
2337 & 4 & Low \\ \hline
12 & 90 & High \\ \hline
\end{ddbasictable}

The accuracy of na{\"i}ve bayes classification was 99.3451\%.

In earlier trials J48 decision tree made
better predictions than logistic regression. After coming across a set of
features that gave us optimum outcome we tried tweaking classification model
arguments. For J48 decision tree we tried changing confidence factor, value of
seed and minimum number of object, but since the accuracy was already high
making these changes to argument did not cause any significant change to the
outcome. Similar results where seen with logistic regression model. J48 and SVM
showed better results, but when we changed threshold of classes J48 produced
better results.

\ddsubsec{res-conc}{Results Conclusion}

The correlation coefficient of temperature, feeling temperature between count
are both 0.40, which shows they have moderate positive linear relationship.
\ddFigref{predicted-vs-actuals} illustrates the variation tendency of
the positively correlated bike counts and temperature by hour.

\ddfigure{\figwidth}{predicted-vs-actuals}{Predicted Values vs. Actual Values}



\ddsubsec{impl}{Implementation}

Data processing, was implemented using Java, that is predictor variable
transformations from nominal to numerical/boolean variables, generation of new
variables, discretizing the variables.

We used Java and Weka API for building and evaluating classification models. We
used Weka API for building J48 Decision trees, Naive Bayes , Logistic
Regression and Support Vector Machines (SVM).

We implemented the regression models using R\ci{team2005r} such as Multivariate
Linear Regression and Support Vector Machines including methods such as
regularization and stepwise regression for model
selection. MASS\ci{rmasspackage} and e1071\ci{dimitriadou2008misc} packages in
R were used for the implementation.


\ddsec{conclusion}{Conclusion}

After comparing results from regression and classification models, we observed
few relations between features and nominal class. As discussed earlier
windspeed does not affect prediction models, since Washington has average
windspeed throughout the year. This assumption was later proved by correlation
table. During step-wise regression the model removed ``winter'' since it did not
have much effect on bike count class. This behavior was observed by
classification models as well.

We used Java and Weka\ci{hall2009weka} API for building and evaluating
classification models. We used Weka API for building J48 Decision trees,
Logistic Regression and Support Vector Machines (SVM).


\ddsec{future-speak}{Future Work}

As a progression to this project, we can add geo-location of each docking
station, duration for each trip to our current data. Often bike sharing systems
give offers where first 30-60 minutes of the ride is free, using this new data
set we will be able to provide better free slots. Using geo-locations of
docking stations we can alert bike sharing companies to restock bikes at a
station.

\bibliography{\bibfile}
\bibliographystyle{plain}

\end{document}
