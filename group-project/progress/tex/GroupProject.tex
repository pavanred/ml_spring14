\documentclass[12pt]{article}

% file admin
\newcommand{\bibfile}{GroupProject}

%\newcommand{\theauthor}{Pavan Reddy <preddy4@uic.edu>, Shreya Ghosh <sghosh21@uic.edu>, Nianzu Ma <nma4@uic.edu>, Paul Landes <plandes2@uic.edu>}
\newcommand{\theauthor}{Pavan Reddy, Shreya Ghosh, Nianzu Ma, Paul Landes}
\newcommand{\doctitle}{CS 491 - Machine Learning - Spring 2014}
\newcommand{\doctype}{Capital BikeShare Data Analysis and Prediction Progress Report}
\newcommand{\theuniversity}{University of Illinois at Chicago}

\newcommand{\ci}[1]{\cite{#1}}
\newcommand{\figwidth}{4in}

\usepackage{zenacademic}

\author{\theauthor}

\begin{document}

\maketitle

\ddsec{intro}{Detailed Objectives}

Bike rentals are affected by various factors. The weather condition in our data set has a lot of impact on the usage of
bicycling. Specifically, usage can be affected by colder weather,
precipitation, and excessive heat. Our research will analyze the effect of
weather, working day, holidays and other factors of bike rentals. We are
currently building machine learning models to predict the number of
bike rental users, both registered and casual users, based on factors such as weather conditions, holidays etc.


\ddsec{progress}{Progress}

Our approach includes using the bike rental data to build a multiple linear
regression model for both daily and hourly data. We built multiple regression
models based on different feature selections.  We performed basic data analysis
to help us choose the features for the regression model. The weather in
Washington DC includes all variations and has an impact on the number of bike
rentals for example the weather situation such as rain has a direct consequence
on the total number of bike rentals. The impact of seasons also affects the
bike rentals. We also see good correlation of the of features such as
temperature and ``feels like'' temperature with the total number of bike
rentals. Based on such data analysis we built multiple linear regression model,
which takes the general form:

\begin{equation}
y_i = \beta_0 + \beta_1x_{i1} + \cdots + \beta_px_{ip} + \varepsilon_i = {\mathbf
x}_i^{T}\beta + \varepsilon_i, \; \; \; i = 1, \ldots, n
\end{equation}

where $p$ is the number of variables.  Examples of values of $x$ are weather
(i.e. rain), temperature, number of rentals, etc.  The $\beta$ values are the
learned parameters.

We built similar regression models using different feature selections for the
daily dataset.

\ddfigure{\figwidth}{temp_counts_cor}{Temperature Bike Counts Correlation}

%\ddfigure{\figwidth}{predicted_vs_actuals}{Predicted Values vs. Actual Values}

We performed 3-fold cross validation to evaluate our model. We also measured
the residuals to estimate the error in the model.

\ddfigure{\figwidth}{residuals_vs_fitted}{Residuals vs. Fitted Values}

%\ddFigref{cross_validation} shows incorrect assumptions about variable dependence.
%\ddfigure{\figwidth}{cross_validation}{Cross Fold Validation}

We built similar regression models for the hourly dataset but we obtained
partial results, we are still working on improving our model based on different
feature selections.

We used R\ci{rproj} to build the regression models and to generate the plots
and we used Java for basic data processing.\newline

\textbf{Action items executed:}

\begin{packedlist}
\item Multiple linear regression for hourly and daily bike rental data. 
\item Different feature selections for regression model of daily data. (Plot)
\item Partial results for hourly bike rental data regression model
\item 3-fold cross validation for bike rentals for daily data regression model
\item Evaluation based on residuals (Residuals vs fitted plot)
\item Implementaion details - R, java for data processing
\end{packedlist}


\ddsec{remaining}{Remaining Work}

We are currently trying to improve the regression models by using the following
methods - 
\begin{packedlist}
\item Experiment and discover better feature selection 
\item Addition of dummy variables e.g. replacing the `weather situation' variable by 4 boolean
variables `clear', `rain', `snow', `cloudy' 
\item Experiment with non linear regression e.g. squares of significant variables in the regression.  
\item Regularization
\end{packedlist}

We are also going to identify the impact of weather conditions, time of day,
day of week, holidays etc. on bike rentals by registered users and casual
users.\newline

We are going to process the data to discretize the total number of bike rentals
and employ classification techniques. We plan to use classification methods
such as Naive Bayes classifier and Logistic Regression and compare the results
of regression and classification.

\ddsubsec{summary}{Summary}

\begin{packedlist}
\item Discretize bike rental data 
\item Use classification models and compare results with regression
\item Experiment and improve regression / classification models based on various feature selections.
\item Process data - add dummy variables (weather situation variable replaced by rainy, fog, snow, sunny variables)
\item Employ regularization
\item Impact on registered users and casual users.
\end{packedlist}

\bibliography{\bibfile}
\bibliographystyle{plain}

\end{document}
