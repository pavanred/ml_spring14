\documentclass[12pt]{article}

% file admin
\newcommand{\bibfile}{GroupProject}

%\newcommand{\theauthor}{Pavan Reddy <preddy4@uic.edu>, Shreya Ghosh <sghosh21@uic.edu>, Nianzu Ma <nma4@uic.edu>, Paul Landes <plandes2@uic.edu>}
\newcommand{\theauthor}{Pavan Reddy, Shreya Ghosh, Nianzu Ma, Paul Landes}
\newcommand{\doctitle}{Machine Learning Project Proposal 2014 Spring}
\newcommand{\doctype}{Capital BikeShare Data Set Analysis and Prediction}
\newcommand{\theuniversity}{University of Illinois at Chicago}

\newcommand{\ci}[1]{\cite{#1}}

\usepackage{zenacademic}

\author{\theauthor}

\begin{document}

\maketitle

\begin{abstract}
Bike sharing systems have made renting bikes efficient and quick with
memberships, multiple bike locations and easy rental and return
process. Through these systems, a user is able to easily rent a bike from a
particular station and return back at another station. Currently, there are
over 500 bike-sharing programs around the world which is composed of over 500
thousands bicycles\ci{ucibikeshare}.
\end{abstract}


\ddsec{intro}{Introduction}

The University of California at Irvine, data set has features like date,
weather, if the day is a weekday or holiday, count of bikes rented, etc. These
features of the data make it apt for research like finding patterns between
different features and number of bikes rented.

In this project we will analyze operational data from the bike-sharing systems,
which was originally generated from Capital Bike Share and derive bike activity
patterns. The problem addressed in this project is finding how many bikes will
be rented based on features given in the dataset. We will use concepts like
regression, kernel function and classification algorithms from Machine Learning
to generate these patterns and then compare their accuracies.

\ddsubsec{team}{Team}
{\bf Pavan Reddy}
CS master student, has experience of machine learning algorithm application and
have research in this domain.

\noindent
{\bf Paul Landes} CS master student, programming expert, has experience of AI
and data visualization.

\noindent
{\bf Shreya Ghosh} CS Ph.D. student, has experience of data mining and NLP.

\noindent
{\bf Nianzu Ma} CS master student, has experience of data mining and NLP.


\ddsec{approach}{Approach}

The Bike Sharing Dataset provides data about the counts of total rental bikes
between years 2011 and 2012 in Capital bikeshare system. The data includes
information about the bike usage of both casual and registered users available
in two resolutions on the time scale; hourly and daily count of rental
bikes. The data also includes the corresponding weather, seasonal, holiday,
time information. The data set is a collation of data from multiple
sources--Capital Bike Sharing system data\ci{capbikeshare}, Weather
information\ci{freemeteo} and holiday schedule\ci{holiday}.

We propose to use this data set to predict the counts of rental bikes based on
weather, seasonal, holidays, temperature, temperature including the wind chill,
wind speeds,

humidity, weather (rain, snow etc.), days (week days, weekends), time of the
day and month of the year. We also propose to predict the usage of rental bikes
for casual users and registered users. And, we propose to predict the usage of
the rental bikes daily and hourly. These predictions can play an important role
in traffic, environmental and health issues management.

Because of the property of the data, we will treat this problem as both
classification problem and regression problem.

Our approach will make use of supervised Machine Learning methods
extensively. The first step is data pre-processing. We first sanitize the
available data, for example handle missing data, if any. Then, the goal is to
identify the best parameters for the regression function. After choosing the
best parameters we build regression models using machine learning algorithms,
Linear Regression Analysis, Support Vector Machines\ci{svm}. We propose to
improve the results as much as possible by using various methods such as
regularization to minimize overfitting the model, which involves estimating the
best regularization parameters. We propose to use ensemble learning methods
such as bootstrap aggregating (bagging), boosting to improve our results.

We also propose to transform the bike rental data to discrete classes because
the value of class attribute changes dramatically of each record. For example,
count of total rental bikes between 0 and 50, and use supervised Machine
Learning classification algorithms such as decision tree, Naive Bayes and
Logistic Regression\ci{logreg}. After best feature selection and building a
classification model, we try to improve our model by using a few methods that
are described earlier. We propose to then use the classification model to
predict the number of rental bikes usage hourly and daily and compare the
results we obtained with classification and regression methods.

Java and the
Weka\footnote{\href{http://www.cs.waikato.ac.nz/ml/weka/}{http://www.cs.waikato.ac.nz/ml/weka/}}
technologies for the implementation of the project.

\ddsec{eval}{Model Evaluation}

Since the total record is more than 17,000, it is big enough for training a
model. So sample a set of training example from hours.csv for training and the
rest of testing. The data will be as 80\% training set and 20\% test set. To
further test the robustness of the classifier, 10-fold cross-validation will be
used for evaluation. Root Mean Squared Error and Mean Absolute Error would be
used as measure for evaluation of regression model\ci{modreg}. Accuracy,
precision, recall and F-score will be used as the measurement for the
evaluation of the classifier.\ci{liu2007web}

\bibliography{\bibfile}
\bibliographystyle{plain}

\end{document}
