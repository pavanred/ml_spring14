\documentclass[12pt]{article}

% file admin
\newcommand{\bibfile}{GroupProject}

%\newcommand{\theauthor}{Pavan Reddy <preddy4@uic.edu>, Shreya Ghosh <sghosh21@uic.edu>, Nianzu Ma <nma4@uic.edu>, Paul Landes <plandes2@uic.edu>}
\newcommand{\theauthor}{Pavan Reddy, Shreya Ghosh, Nianzu Ma, Paul Landes}
\newcommand{\doctitle}{Machine Learning Project Proposal 2014 Spring}
\newcommand{\doctype}{Capital BikeShare Data Set Analysis and Prediction}
\newcommand{\theuniversity}{University of Illinois at Chicago}

\newcommand{\ci}[1]{\cite{#1}}

\usepackage{zenacademic}

\author{\theauthor}

\begin{document}

\maketitle


\ddsec{intro}{Introduction}

Bike sharing systems have made renting bikes efficient and quick with
memberships, multiple bike locations and easy rental and return
process. Through these systems, a user is able to easily rent a bike from a
particular station and return back at another station. Currently, there are
over 500 bike-sharing programs around the world which is composed of over 500
thousands bicycles\ci{ucibikeshare}.

\ddsubsec{obj}{Objectives}

The University of California at Irvine, data set has features like date,
weather, weekdays, holidays and count of bikes rented. These features of the
data make it apt for research such as finding patterns between different
features and number of bikes rented.

In this project we will analyze operational data from the bike-sharing systems,
which was originally generated from Capital Bike Share and derive bike activity
patterns. The problem addressed in this project is finding how many bikes will
be rented based on features given in the dataset. We will use concepts like
regression, kernel function and classification algorithms to generate these
patterns and then compare their accuracies.

\ddsubsec{team}{Team}
{\bf Pavan Reddy} CS master student, has experience of machine learning
algorithm application and have research in this domain.

\noindent
{\bf Paul Landes} CS master student, programming expert, has experience of AI
and data visualization.

\noindent
{\bf Shreya Ghosh} CS Ph.D. student, has experience of data mining and NLP.

\noindent
{\bf Nianzu Ma} CS master student, has experience of data mining and NLP.


\ddsec{approach}{Approach}

The Bike Sharing Dataset provides data about the counts of total rental bikes
between years 2011 and 2012 for the Capital bikeshare system. The data includes
information about the bike usage of both casual and registered users available
in two resolutions on the time scale; hourly and daily count of rental
bikes. The data also includes the corresponding weather, seasonal, holiday,
time information. The data set is a collation of data from multiple
sources--Capital Bike Sharing system data\ci{capbikeshare}, Weather
information\ci{freemeteo} and holiday schedule\ci{holiday}.

We propose to use this data set to predict the counts of rental bikes based on:

\begin{itemize}
\item Seasonal holidays
\item Weather: wind chill, wind speeds, humidity, precipitation
\item Calendar: week day vs. weekend, time of the day and month of the
      year
\end{itemize}

Predictions will include not only the usage of rental bikes for casual users
and registered users but also the usage of the rental bikes daily and
hourly. These predictions can play an important role in traffic, environmental
and health issues management.  The problem will be treated as both a
classification problem and regression problem.

Our approach will make use of supervised machine learning methods,
specifically:

\begin{enumerate}
\item The first step is data pre-processing: sanitize the available
      data and handle any missing data.

\item Identify the best parameters for the regression function: after choosing
      the best parameters we will build regression models using linear
      regression analysis and support vector machines\ci{svm}.

\item Post-processing to improve the results using various methods such as
      regularization to minimize over-fitting the model, which involves
      estimating the best regularization parameters.

\item Improve results further with ensemble learning methods such as bootstrap
      aggregating (bagging).
\end{enumerate}

We will transform the bike rental data to discrete classes as the value of
class attribute changes dramatically of each record.  For example, using the
discredited counts of total rental bikes ranges between 0 and 50 in conjunction
with supervised learning classification algorithms such as decision tree, naive
bayes and logistic regression\ci{logreg}.  Post-processing data improvements
will be employed after best feature selection and the building of a
classification model.  The classification model will then be used to predict
the number of rental bikes usage hourly and daily and compare the results we
obtained with classification and regression methods.

Java and the
Weka\footnote{\href{http://www.cs.waikato.ac.nz/ml/weka/}{http://www.cs.waikato.ac.nz/ml/weka/}}
technologies for the implementation of the project.


\ddsec{eval}{Model Evaluation}

Since the total record is more than 17,000, it is big enough for training a
model. So sample a set of training example from hours.csv for training and the
rest of testing. The data will be as 80\% training set and 20\% test set. To
further test the robustness of the classifier, 10-fold cross-validation will be
used for evaluation. Root Mean Squared Error and Mean Absolute Error would be
used as measure for evaluation of regression model\ci{modreg}. Accuracy,
precision, recall and F-score will be used as the measurement for the
evaluation of the classifier.\ci{liu2007web}

\bibliography{\bibfile}
\bibliographystyle{plain}

\end{document}
