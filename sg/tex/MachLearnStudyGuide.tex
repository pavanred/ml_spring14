\documentclass[12pt]{article}

% ------ variables to modify for each doc instance
\newcommand{\theauthor}{Paul Landes}
\newcommand{\revision}{1.0}
\newcommand{\doctitle}{Machine Learning Spring 2014}
\newcommand{\doctype}{Study Guide}
\newcommand{\theuniversity}{University of Illinois at Chicago}
% ------

\usepackage{zenacademic}
\usepackage{zenacademicstats}
\ddfontset

\begin{document}

\maketitle
\listofddequations
\thispagestyle{empty}

\begin{sgequationsite}{mean}{Mean}{http://www.khanacademy.org/math/cc-seventh-grade-math/cc-7th-probability-statistics/cc-7th-central-tendency/v/mean-median-and-mode}{Khan's Academy}
\mu \df \sum x P(x) = \frac{1}{N}\sum_{i=1}^{N} x_i
\end{sgequationsite}

\begin{sgequationsite}{var-std}{Variance and Standard Deviation}{http://www.khanacademy.org/math/probability/descriptive-statistics/variance_std_deviation/v/range--variance-and-standard-deviation-as-measures-of-dispersion}{Khan's Academy}
\textrm{Var}(X) = E[(X - \mu)^2] \df \frac{1}{N}\sum_{i=1}^{N}(x_i - \mu)^2\\
\textrm{Var}(X) = \sigma^2 \;
\Longleftrightarrow \;
\sigma = \sqrt{\textrm{Var}(X)}\\
\Longleftrightarrow \;
\int(x - \mu)^2 f(x)dx = \int x^2 f(x)dx - \mu^2
\end{sgequationsite}

\begin{sgequationpagedesc}{likelihood-book}{Likelihood Per Book}{67}{%
Size principle/Occam's razor: model favors simplest hypothesis consistent with
data.}
P(\td \g h) = \left[ \frac{1}{|h|} \right]^N
\end{sgequationpagedesc}

\begin{sgequationpagedesc}{likelihood}{Likelihood Function}{Slides PS 2 19}{%
Probability of data given this model hypothesis.}
\likelihoodfn
\end{sgequationpagedesc}

\begin{sgequationdesc}{mle}{Maximum Likelihood Estimation (MLE)}{Slides PS 2 22}{%
Choose parameters that make the data most probable.}
\likelihoodest = \argmax_{\theta}P(\td \g \theta)
\end{sgequationdesc}

\begin{sgequationdesc}{map}{Maximum a Posteriori Estimate (MAP)}{Slides PS 2 37}{%
Somewhat Bayesian (``point + prior'').
\begin{packedlist}
\item Choose a likelihood function $\likelihoodfn$
\item Choose a prior parameter dist $P(\theta)$
\item Use MAP.
\item Make predictions using MAP: $P(\td' \g \mapfn)$
\end{packedlist}}
\mapfn = \argmax_{\theta}P(\td \g \theta) P(\theta)
\end{sgequationdesc}

\begin{ddexample}{Likelihood and MLE Example}{Office Hours}
\noindent
\begin{math}
P(\theta) = \left\{
\begin{array}{l l}
\frac{1}{3} & \theta = \theta_1\\
\frac{1}{3} & \theta = \theta_2\\
\frac{1}{3} & \theta = \theta_3
\end{array} \right.\\
\end{math}

\noindent
\begin{math}
\td = \{ H, H \} \;\; \text{(two heads)}\\
\theta_1 = .5\\
\theta_2 = .3\\
\theta_3 = .8\\
\likelihood_1 =.5 \cdot .5 = .25
\end{math}

\vspace{.2cm}

\noindent
Probability of two heads (H H) using likelihood and MAP estimates:\\
\begin{tabular}{c|c|l}
$N$ & $\likelihood_N$ & $\likelihoodfn P(\theta)$ (MAP)\\
\hline
$1$ & $.5 \cdot .5 = .25$ & $.25 \cdot \frac{1}{3} = .083$\\
$2$ & $.3 \cdot .3 = .09$ & $.09 \cdot \frac{1}{3} = .03$\\
$3$ & $.8 \cdot .8 = .64$ & $.64 \cdot \frac{1}{3} = .213$\\
\end{tabular}

\vspace{.2cm}
\noindent
Parameter $\theta_3$ wins with an MLE $\likelihoodest = .64$.\\
Parameter $\theta_3$ wins with an MAP $\mapfn = .213$.

\vspace{.2cm}

\noindent
Bayes:\\
\begin{displaymath}
\begin{gathered}
P(\theta \g \td) = \frac{P(\td \g \theta)P(\theta)}{P(\td)}=
\frac{P(\td \g \theta)P(\theta)}{\sum_{\theta'}P(\td \g \theta')P(\theta')}=\\
P(H \g \td) = \sum_{\theta'\in \{\theta_1 \theta_2, \theta_3\}}P(\theta' \g \td)P(H \g \td)=\\
\frac
{.083 \cdot .5 + .03 \cdot .3 + .213 \cdot .8}
{.083 + .03 + .213}=.67
\end{gathered}
\end{displaymath}

\end{ddexample}

\begin{sgequationpagedesc}{bindist}{Bernoulli/Binomial Distribution}{34}{%
Models outcomes of coin tosses.  Coin toss $n$ times with $X \in \{0, \dots,
n\}$.  Probability $\theta$ is the parameter of, for example, fairness of the
coin.

\begin{packedlist}
  \item Expected number of ``heads'' outcomes in 10 flips.
  \item Given 10 flips what is the probability of at least 8 being ``heads''?
  \item Given 3 heads, 2 tails, what is the estimate of $\theta$?
  \item How far will this estimate be from true $\theta^*$?
\end{packedlist}

\noindent
Bernoulli is the case where $n = 1$ (that is, one coin toss).}
\textrm{Bin}(k \g n, \theta) \df {n\choose k} \theta^{k}(1 - \theta)^{n - k}
\end{sgequationpagedesc}

\begin{sgequationpagedesc}{bincof}{Binomial Coefficient}{34}{%
Binomial coefficients are a family of positive integers that occur as
coefficients in the binomial theorem.}
{n \choose k} = \frac{n!}{k!(n - k)!}
\end{sgequationpagedesc}

\begin{sgequationpagedesc}{multdist}{Multinomial/Multinouli Distribution}{35}{%
Models the outcomes of tossing a $K$-sided die.  Let $\textbf{x} = (x_1 \dots
x_K)$ be a randome vector, where $x_j$ is the number of times side $j$ of the
die occurs.}
\textrm{Mu}(\textbf{x} \g n, \theta) \df
{n \choose {x_1 \dots x_K}} \prod_{j=1}^K\theta_{j}^{x_j}\\
\textrm{$\theta_j$ is the probability that side $j$ shows up, and}\\
{n \choose {x_1 \dots x_K}} \df \frac{n!}{x_1!x_2! \dots x_K!}
\end{sgequationpagedesc}

\concept{Poisson/Empirical Distribution}{37}

\begin{sgequationpage}{gaussian-dist}{Gaussian (Normal) Distribution}{38}
P(x) \df \Nu(x \g \mu, \sigma^2) =
\frac{1}{{\sigma \sqrt {2\pi } }}e^{{{ - \left( {x - \mu } \right)^2 } \mathord{\left/ {\vphantom {{ - \left( {x - \mu } \right)^2 } {2\sigma ^2 }}} \right. \kern-\nulldelimiterspace} {2\sigma ^2 }}}
\end{sgequationpage}

\begin{sgequationpagedesc}{beta-dist}{Beta Distribution}{42}{%
Where $\Gamma$ is the conjugate prior.}
\textrm{Beta}(x \g a, b) = \frac{1}{B(a,b)^{x^{a-1}(1 - x)^{b - 1}}}\\
B \df \frac{\Gamma(a)\Gamma(b)}{\Gamma(a + b)}
\end{sgequationpagedesc}

\begin{sgequation}{product-rule}{Product Rule}{ES Class}
P(a \a b) = P(a \g b) P(b) \Leftrightarrow\\
P(X = x, Y = y) = P(X = x \g Y = y) \cdot P(Y = y)
\end{sgequation}

\begin{sgequationpage}{bayes-rule}{Bayes Rule}{29}
P(X = x \g Y = y) = \frac{P(X = x, Y = y)}{P(Y = y)} =
\frac{P(X = x)P(Y = y \g X = x)}{\sum_{x'}P(X = x')P(Y = y \g X = x')}
\end{sgequationpage}

\end{document}
